\documentclass[a4paper,11pt]{article}
\usepackage{mathtools}

\DeclareMathOperator{\xder}{\, \mathrm{d}x}
\DeclareMathOperator{\tder}{\, \mathrm{d}t}
\DeclareMathOperator{\arccsc}{arccsc}
\DeclareMathOperator{\arcsec}{arcsec}
\DeclareMathOperator{\argsh}{argsh}
\DeclareMathOperator{\argch}{argch}
\DeclareMathOperator{\argth}{argth}

\begin{document}
	\section{Métodos de Integración}
		\subsection{Cambio de variable}
			$ \int f(x) \xder = \Big{/}
				\begin{array}{c}
					x = \phi (t) \\
					dx = \phi'(t) \tder
				\end{array}
				\Big{/}
				= \int f(\phi(t)) \phi' (t) \tder =
				\int g(t) \tder = G (t) + C = G (\phi^{-1} (x)) + C
			$
		\subsection{Por partes}
			$ \int u \, dv = uv - \int v \, du $

			\subsubsection{$ \int R (Pm(x), e^x) \xder$}
				$ u = Pm(x) $ \quad 
				Ej: $ \int (2x + 1) e^x \xder $
			\subsubsection{$ \int R (Pm(x), \cos ax, \sin bx) \xder $}
				$ u = Pm(x) $ \quad
				Ej: $ \int (2x + 1) \cos x \xder $
			\subsubsection{$ \int R (e^{\alpha x}, \sin ax, \cos bx) \xder $}
				$ u = e^{\alpha x} $ ó $ u = \sin ax, \cos bx $ \quad
				Ej: $ \int e^x \cos x \xder $

				Éste tipo se hace dos veces por partes, podemos llamar 
				\textit{u} a la exponencial o ala trigonométrica, pero la
				segunda vez estamos obligadxs a elegir lo mismo. Si está bien
				hecha al hacer por partes la segunda vez, nos vuelve a salir
				la del enunciado.
			\subsubsection{$ \int (\ln, \arcsin, \arccos, \arctan, \argsh, \argch, \argth ) \xder $}
				$ ( \dots ) = u $
		\subsection{Racionales}
			$ \int \frac{P(x)}{Q(x)} \xder =
				\Big{/}
					asdf
				\Big{/}
				= \int [ C(x) + \frac{R(x)}{Q(x)} ] \xder =
				\int C(x) \xder + \int \frac{R(x)}{Q(x)} \xder \leftarrow \text{Pasos} $

				Pasos
			\begin{enumerate}
				\item factorizar Q(x): $ Q(x) = 0 \rightarrow
					\big{ \{ } 
						x = a, 
						x = b, 
						\dots,
						x = n 
					\big{ \} } $
				\item descomponer en 
			\end{enumerate}
\end{document}
