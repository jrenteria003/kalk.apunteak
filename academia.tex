\documentclass[8pt]{article}
\usepackage[a4paper,margin=2cm]{geometry}
\usepackage{mathtools}

\DeclareMathOperator{\xder}{\, \mathrm{d}x}
\DeclareMathOperator{\tder}{\, \mathrm{d}t}
\DeclareMathOperator{\arccsc}{arccsc}
\DeclareMathOperator{\arcsec}{arcsec}
\DeclareMathOperator{\argsh}{argsh}
\DeclareMathOperator{\argch}{argch}
\DeclareMathOperator{\argth}{argth}

\begin{document}
	\section{Métodos de Integración}
		\subsection{Cambio de variable}
			$ \int f(x) \xder = \Big{/}
				\begin{array}{c}
					x = \phi (t) \\
					dx = \phi'(t) \tder
				\end{array}
				\Big{/}
				= \int f(\phi(t)) \phi' (t) \tder =
				\int g(t) \tder = G (t) + C = G (\phi^{-1} (x)) + C
			$
		\subsection{Por partes}
			$ \int u \, dv = uv - \int v \, du $

			\subsubsection{$ \int R (Pm(x), e^x) \xder$}
				$ u = Pm(x) $ \quad 
				Ej: $ \int (2x + 1) e^x \xder $
			\subsubsection{$ \int R (Pm(x), \cos ax, \sin bx) \xder $}
				$ u = Pm(x) $ \quad
				Ej: $ \int (2x + 1) \cos x \xder $
			\subsubsection{$ \int R (e^{\alpha x}, \sin ax, \cos bx) \xder $}
				$ u = e^{\alpha x} $ ó $ u = \sin ax, \cos bx $ \quad
				Ej: $ \int e^x \cos x \xder $

				Éste tipo se hace dos veces por partes, podemos llamar 
				\textit{u} a la exponencial o ala trigonométrica, pero la
				segunda vez estamos obligadxs a elegir lo mismo. Si está bien
				hecha al hacer por partes la segunda vez, nos vuelve a salir
				la del enunciado.
			\subsubsection{$ \int (\ln, \arcsin, \arccos, \arctan, \argsh, \argch, \argth ) \xder $}
				$ ( \dots ) = u $
		\subsection{Racionales}
			$ \int \frac{P(x)}{Q(x)} \xder =
				\Big{/}
					asdf
				\Big{/}
				= \int [ C(x) + \frac{R(x)}{Q(x)} ] \xder =
				\int C(x) \xder + \int \frac{R(x)}{Q(x)} \xder \leftarrow \text{Pasos} $

				Pasos
			\begin{enumerate}
				\item factorizar Q(x): $ Q(x) = 0 \Rightarrow
					\Bigg{ \{ } 
						x = a, 
						x = b, 
						\dots,
						x = n 
					\Bigg{ \} } $
				\item descomponer $ \frac{R(x)}{Q(x)} $ en fracciones simples:
					\begin{itemize}
						\item $ \frac{R(x)}{Q(x)} = \frac{A}{x-a} + \frac{B}{x-b}
							+ \dots + \frac{N}{x-n}
							\begin{array}{c}
								\text{Común}\\
								\Rightarrow\\
								\text{denominador}
							\end{array}
							\Bigg \{ %} don't tell your editor about this!
							\begin{array}{c}
								\text{Identificar}\\
								\text{ó}\\
								\text{dar valores}
							\end{array}
							\Rightarrow
							\begin{array}{c}
								\text{Resolver}\\
								\text{el}\\
								\text{sistema}
							\end{array} \Rightarrow \text{A, B, } \dots \text{N}$

						\item Soluciones Reales Múltiples: $ Q(x) = (x-a)^p (x-b)^q $
							$ \frac{P(x)}{Q(x)} = \frac{A_1}{x-a} +
							\frac{A_2}{(x-a)^2} + \frac{A_3}{(x-a)^3} + \dots +
							\frac{A_p}{(x-a)^p} + \frac{B_1}{(x-b)} + \dots +
							\frac{B_q}{(x-b)^q} 
							\begin{array}{c}
								\text{Común}\\
								\Rightarrow\\
								\text{denominador}
							\end{array}
							\Bigg \{
							\begin{array}{c}
								\text{Identificar}\\
								\text{ó}\\
								\text{dar valores}
							\end{array}
							\Rightarrow
							\begin{array}{c}
								\text{Resolver}\\
								\text{el}\\
								\text{sistema}
							\end{array} \Rightarrow 
							\begin{array}{l}
								A_1, A_2, \dots A_p\\
								B_1, B_2, \dots A_q
							\end{array}$

						\item Soluciones Imaginarias Simples: $ Q(x) = \dots (a x^2 + b x + c) $
							$ \frac{R(x)}{Q(x)} = \dots +
							\frac{Mx + N}{a x^2 + b x + c}
							\begin{array}{c}
								\text{Común}\\
								\Rightarrow\\
								\text{denominador}
							\end{array}
							\Bigg \{
							\begin{array}{c}
								\text{Identificar}\\
								\text{ó}\\
								\text{dar valores}
							\end{array}
							\Rightarrow
							\begin{array}{c}
								\text{Resolver}\\
								\text{el}\\
								\text{sistema}
							\end{array} \Rightarrow M, N $
						\item Soluciones Imaginarias Múltiples $ \Rightarrow $ \underline{Hermite}
					\end{itemize} 

				\item Integrar las fracciones simples
					\begin{itemize}
						\item S.R.S: $ \Rightarrow \int \frac{1}{x-a} \xder = \ln |x-a| + C $
							% SRS xd
						\item S.R.M: $ \Rightarrow \int \frac{1}{(x-a)^n} \xder = \frac{(x-a)^{-n+1}}{-n+1} + C $
						\item S.I.S: $ \Rightarrow \int \frac{Mx+N}{a x^2 + b x + c} \xder = $
					\end{itemize}

			\end{enumerate}
		\subsubsection{Hermite S.I.M}
			$ \int \frac{P(x)}{Q(x)} \xder = \frac{p_2 (x)}{q_2 (x)} + \int \frac{p_1 (x)}{q_1 (x)} \xder $

			$ q_1 (x) $ : polinomio formado por los factores de $ Q(x) $ elevados a uno

			$ p_1 (x) $ : polinomio completo de coeficientes indeterminados y de grado uno menos que $ q_1 (x) $

			$ q_2 (x) $ : $ \frac{Q(x)}{q_1 (x)} $

			$ p_2 (x) $ : polinomio completo de coeficientes indeterminados y de grado uno menos que $ q_2 (x) $

			$ \frac{d}{\xder} \Rightarrow \frac{P(x)}{Q(x)} = \frac{d}{\xder} 
				\Big[ \frac{p_2 (x)}{q_2 (x)} \Big] + \frac{p_1 (x)}{q_2 (x)}
				\begin{array}{c}
					\text{Común}\\
					\Rightarrow\\
					\text{denominador}
				\end{array}
				\Bigg \{
				\begin{array}{c}
					\text{Identificar}\\
					\text{ó}\\
					\text{dar valores}
				\end{array}
				\Rightarrow
				\begin{array}{c}
					\text{Resolver}\\
					\text{el}\\
					\text{sistema}
				\end{array} \Rightarrow p_1 , p_2 $

				$ \int \frac{p_1 (x)}{q_1 (x)} \xder \Rightarrow $ fracciones simples

	\subsection{Exponenciales}
		$ \int R (e^x) \xder $

			cambio $ \Rightarrow
			\begin{array}{c}
				e^x = t\\
				e^x \xder = \tder
			\end{array}
			\Rightarrow $ \underline{Racional}

\end{document}
