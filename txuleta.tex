\documentclass[8pt]{article}
\usepackage[a4paper,margin=1cm]{geometry}
\usepackage{mathtools}

\DeclareMathOperator{\xder}{\, \mathrm{d}x}
\DeclareMathOperator{\tder}{\, \mathrm{d}t}
\DeclareMathOperator{\zder}{\, \mathrm{d}z}
\DeclareMathOperator{\arccsc}{arccsc}
\DeclareMathOperator{\arcsec}{arcsec}
\DeclareMathOperator{\argsh}{argsh}
\DeclareMathOperator{\argch}{argch}
\DeclareMathOperator{\argth}{argth}

\begin{document}
	\section{Métodos de Integración}
		\subsection{Aldagai aldaketa}
			$ \int f(x) \xder = \Big{/}
				\begin{array}{c}
					x = \phi (t) \\
					dx = \phi'(t) \tder
				\end{array}
				\Big{/}
				= \int f(\phi(t)) \phi' (t) \tder =
				\int g(t) \tder = G (t) + C = G (\phi^{-1} (x)) + C
			$
		\subsection{Zatika}
			$ \int u \, dv = uv - \int v \, du $

			\subsubsection{$ \int R (Pm(x), e^x) \xder$}
				$ u = Pm(x) $ \quad 
				Adb.: $ \int (2x + 1) e^x \xder $
			\subsubsection{$ \int R (Pm(x), \cos ax, \sin bx) \xder $}
				$ u = Pm(x) $ \quad
				Adb.: $ \int (2x + 1) \cos x \xder $
			\subsubsection{$ \int R (e^{\alpha x}, \sin ax, \cos bx) \xder $}
				$ u = e^{\alpha x} $ ó $ u = \sin ax, \cos bx $ \quad
				Adb.: $ \int e^x \cos x \xder $

				Mota hau birritan egiten da, \textit{u} exponentziala ala
				trigonometrikoa izan daiteke; bigarren aldiz egiterakoan
				lehen aukeratutakoa berriro ere aukeratuko beharko genuke.

			\subsubsection{$ \int (\ln, \arcsin, \arccos, \arctan, \argsh, \argch, \argth ) \xder $}
				$ ( \dots ) = u $
		\subsection{Arrazionalak}
			$ \int \frac{P(x)}{Q(x)} \xder =
				\Big{/}
					asdf
				\Big{/}
				= \int [ C(x) + \frac{R(x)}{Q(x)} ] \xder =
				\int C(x) \xder + \int \frac{R(x)}{Q(x)} \xder \leftarrow \text{Urratsak} $

				Urratsak
			\begin{enumerate}
				\item faktorizatu Q(x): $ Q(x) = 0 \Rightarrow
					\Bigg{ \{ } 
						x = a, 
						x = b, 
						\dots,
						x = n 
					\Bigg{ \} } $
				\item deskonposatu $ \frac{R(x)}{Q(x)} $ zatiki simpleetan:
					\begin{itemize}
						\item $ \frac{R(x)}{Q(x)} = \frac{A}{x-a} + \frac{B}{x-b}
							+ \dots + \frac{N}{x-n}
							\begin{array}{c}
								\text{Zatitzaile}\\
								\Rightarrow\\
								\text{komuna}
							\end{array}
							\Bigg \{ %} don't tell your editor about this!
							\begin{array}{c}
								\text{Identifikatu}\\
								\text{edo}\\
								\text{balioak eman}
							\end{array}
							\Rightarrow
							\begin{array}{c}
								\text{Sistema}\\
								\text{ebatzi}
							\end{array} \Rightarrow \text{A, B, } \dots \text{N}$

						%???\item Soluzio Erreal Konplexsuak: $ Q(x) = (x-a)^p (x-b)^q $
						\item Soluzio Erreal Anitzak: $ Q(x) = (x-a)^p (x-b)^q $
							$ \frac{P(x)}{Q(x)} = \frac{A_1}{x-a} +
							\frac{A_2}{(x-a)^2} + \frac{A_3}{(x-a)^3} + \dots +
							\frac{A_p}{(x-a)^p} + \frac{B_1}{(x-b)} + \dots +
							\frac{B_q}{(x-b)^q} 
							\begin{array}{c}
								\text{Zatitzaile}\\
								\Rightarrow\\
								\text{komuna}
							\end{array}
							\Bigg \{
							\begin{array}{c}
								\text{Identifikatu}\\
								\text{edo}\\
								\text{balioak eman}
							\end{array}
							\Rightarrow
							\begin{array}{c}
								\text{Sistema}\\
								\text{ebatzi}
							\end{array} \Rightarrow 
							\begin{array}{l}
								A_1, A_2, \dots A_p\\
								B_1, B_2, \dots A_q
							\end{array}$

						\item Soluzio Irudikari Simpleak: $ Q(x) = \dots (a x^2 + b x + c) $
							$ \frac{R(x)}{Q(x)} = \dots +
							\frac{Mx + N}{a x^2 + b x + c}
							\begin{array}{c}
								\text{Zatitzaile}\\
								\Rightarrow\\
								\text{komuna}
							\end{array}
							\Bigg \{
							\begin{array}{c}
								\text{Identifikatu}\\
								\text{edo}\\
								\text{balioak eman}
							\end{array}
							\Rightarrow
							\begin{array}{c}
								\text{Sistema}\\
								\text{ebatzi}
							\end{array} \Rightarrow M, N $
						\item Soluzio Irudikaria Anitzak$ \Rightarrow $ \underline{Hermite}
					\end{itemize} 

				\item Zatiki simpleak integratu
					\begin{itemize}
						\item S.E.S: $ \Rightarrow \int \frac{1}{x-a} \xder = \ln |x-a| + C $
							% SRS xd ( gaztelerazko bertsioan )
						\item S.E.A: $ \Rightarrow \int \frac{1}{(x-a)^n} \xder = \frac{(x-a)^{-n+1}}{-n+1} + C $
						\item S.E.S: $ \Rightarrow \int \frac{Mx+N}{a x^2 + b x + c} \xder = $
					\end{itemize}

			\end{enumerate}
		\subsubsection{Hermite S.I.A}
			$ \int \frac{P(x)}{Q(x)} \xder = \frac{p_2 (x)}{q_2 (x)} + \int \frac{p_1 (x)}{q_1 (x)} \xder $

			$ q_1 (x) $ : polinomio formado por los factores de $ Q(x) $ elevados a uno

			$ p_1 (x) $ : polinomio completo de coeficientes indeterminados y de grado uno menos que $ q_1 (x) $

			$ q_2 (x) $ : $ \frac{Q(x)}{q_1 (x)} $

			$ p_2 (x) $ : polinomio completo de coeficientes indeterminados y de grado uno menos que $ q_2 (x) $

			$ \frac{d}{\xder} \Rightarrow \frac{P(x)}{Q(x)} = \frac{d}{\xder} 
				\Big[ \frac{p_2 (x)}{q_2 (x)} \Big] + \frac{p_1 (x)}{q_2 (x)}
				\begin{array}{c}
					\text{Zatitzaile}\\
					\Rightarrow\\
					\text{komuna}
				\end{array}
				\Bigg \{
				\begin{array}{c}
					\text{Identifikatu}\\
					\text{edo}\\
					\text{balioak eman}
				\end{array}
				\Rightarrow
				\begin{array}{c}
					\text{Sistema}\\
					\text{ebatzi}
				\end{array} \Rightarrow p_1 , p_2 $

				$ \int \frac{p_1 (x)}{q_1 (x)} \xder \Rightarrow $ Zatiki sinpleak

	\subsection{Exponentzialak}
		$ \int R (e^x) \xder $

			aldaketa $ \Rightarrow
			\begin{array}{c}
				e^x = t\\
				e^x \xder = \tder
			\end{array}
			\Rightarrow $ \underline{Arrazionala}

		\subsubsection{Irracionalak}
			\begin{itemize}
				\item $ \int R (x, x^{r/s}, \dots , x^{p/q}) \xder $\\
					aldaketa $ \Rightarrow x = z^{\mu},
					\mu = \text{M.C.M} (s, \dots, q) \Rightarrow Arrazionala $
				\item $ \int R (x, (\frac{ax+b}{cx+d})^{r/s}, \dots, 
					(\frac{ax+b}{cx+d})^{p/q}) \xder $\\
					aldaketa $ \Rightarrow (\frac{ax+b}{cx+d}) = z^{\mu},
					\mu = \text{MKT} (s, \dots, q) \Rightarrow Racional $
				\item aldaketa trigonometrikoa edo hiperbolikoa:\\
					$ \sqrt{a^2 - [f(x)]^2} \Rightarrow f(x) = a \sin t $\\
					$ \sqrt{[f(x)]^2 - a^2} \Rightarrow f(x) = a \cosh t $\\
					$ \sqrt{[f(x)]^2 + a^2} \Rightarrow f(x) = a \sinh t $
				\item kasu bereziak:
					\begin{enumerate}
						\item Metodo Alemaniarra:
							$ \int \frac{P(x)}{\sqrt{a x^2 + bx + c}} \xder =
							q(x) \sqrt{a x^2 + bx + c} + k \int \frac{\xder}{\sqrt{a x^2 + bx + c}} $\\

							$ q(x) $: polinomio completo de coeficientes indeterminados y
							de grado uno menos que $ P(x) $\\
							$ k $: constante
								
							$ \frac{\mathrm{d}}{\xder} \Rightarrow 
							\frac{P(x)}{\sqrt{a x^2 + bx + c}} = 
							\frac{\mathrm{d}}{\xder}
							[q(x) \sqrt{a x^2 + bx + c} + \frac{k}{\sqrt{a x^2 + bx + c}}
							\begin{array}{c}
								\text{Zatitzaile}\\
								\Rightarrow\\
								\text{komuna}
							\end{array}
							\Bigg \{
							\begin{array}{c}
								\text{Identifikatu}\\
								\text{edo}\\
								\text{balioak eman}
							\end{array}
							\Rightarrow
							\begin{array}{c}
								\text{Sistema}\\
								\text{ebatzi}
							\end{array} \Rightarrow q(x), k $
							$ \quad \int \frac{\xder}{\sqrt{a x^2 + bx + c}}
							\Rightarrow
							\begin{array}{c}
								\arcsin\\
								\argsh\\
								\argch
							\end{array} $
						\item Bigarren kasu berezia:
							$ \int \frac{\xder}{(x - \alpha)^n \sqrt{a x^2 + bx + c}} $\\
							aldaketa $ \Rightarrow 
							\begin{array}{c}
								x - \alpha = \frac{1}{z}\\
								\xder = \frac{-1}{z^2} \zder
							\end{array}
							\Rightarrow $ \underline{Arrazionala}
					\end{enumerate}
			\end{itemize}

		\subsubsection{Binomia: $ \int x^m (a + b x^n)^p \xder $}
			\begin{enumerate}
				\item $ p \in \Z \Rightarrow $ 
					\underline{Lehenengo motako irrazionala}
					$ \Rightarrow $
					\underline{Arrazionala}
				\item $ \frac{m+1}{n} \in \Z \Rightarrow 
					\begin{array}{c}
						x^n = t\\
						x = t^{\frac{1}{n}}\\
						\xder = \frac{1}{n} t^{\frac{1}{n}-1} \tder
					\end{array} \Rightarrow
					\int t^{\frac{m}{n} (a + bt)^p \frac{1}{n}
					t^{\frac{1}{n}-1} \tder = 
					\frac{1}{n} \int t^{\frac{m+1}{n}-1}
					(a + bt)^p \tder =
					\Big/
					\begin{array}{c}
						a + bt = z\\
						b \tder = \zder
					\end{array} 
					\Big/ = 
					\frac{1}{nb} \int (\frac{z-a}{b})^{\frac{m+1}{n}-1}
					z^p \zder \Rightarrow $
					\underline{Lehenengo motako irrazionala} 
					$ \Rightarrow $
					\underline{Arrazionala}
				\item $ \frac{m+1}{n} + p \in \Z \Rightarrow
					\begin{array}{(c)}
						x^n = t\\
						x = t^{\frac{1}{n}}\\
						\xder = \frac{1}{n} t^{\frac{1}{n}-1} \tder
					\end{array} \Rightarrow
					\int t^{\frac{m}{n}} (a + bt)^p \frac{1}{n}
					t^{\frac{1}{n}-1} \tder = \frac{1}{n}
					\int t^{\frac{m+1}{n}-1} (a + bt)^p
					\frac{t^p}{t^p} \tder = \frac{1}{n}
					\int t^{\frac{m+1}{n}-1+p}
					(\frac{a + bt}{t})^p \tder \Rightarrow
					\Big/
					\begin{array}{c}
						\frac{a + bt}{t} = z\\
						\frac{-a}{t^2} \tder = \zder
					\end{array} \Big/
					$ \underline{Lehenengo motako irrazionala}
					$ \Rightarrow $ \underline{Arrazionala}
			\end{enumerate}

	\section{Integral Mugatua}
		\subsection{Barrowren erregela}
			$ \int_a_b f(x) \xder = [f(x)]{_a ^b} = f(b) - f(a) $

\end{document}
