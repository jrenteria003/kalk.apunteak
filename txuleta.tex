\documentclass[8pt]{article}
\usepackage[a4paper,margin=1cm]{geometry}
\usepackage{mathtools}
\setlength{\parindent}{0cm} % Default is 15pt.

\DeclareMathOperator{\xder}{\, \mathrm{d}x}
\DeclareMathOperator{\tder}{\, \mathrm{d}t}
\DeclareMathOperator{\zder}{\, \mathrm{d}z}
\DeclareMathOperator{\arccsc}{arccsc}
\DeclareMathOperator{\arcsec}{arcsec}
\DeclareMathOperator{\argsh}{argsh}
\DeclareMathOperator{\argch}{argch}
\DeclareMathOperator{\argth}{argth}
\DeclareMathOperator{\cosec}{cosec}
\DeclareMathOperator{\cotan}{cotan}

\begin{document}
	
	\section{Berehalakoak}

		$ \int k f(x) \xder = k \int f(x) \xder $

		$ \int [f(x) + g(x)] \xder = \int f(x) \xder + \int g(x) \xder $

		$ \int f\prime(x) {[ f(x) ]}^n \xder = \frac{{[f(x)]}^{n+1}}{n+1} \quad baldin n \neq -1 $

		$ \int \frac{f\prime(x)}{f(x)} \xder = \ln | f(x) | + k $

		$ \int f\prime(x) e^{f(x)} \xder = e^{f(x)} + k $

		$ \int f\prime(x) a^{f(x)} \xder = \frac{a^{f(x)}}{\ln a} + k $

		$ \int f\prime(x) \sin f(x) \xder = - \cos f(x) + k $

		$ \int f\prime(x) \cos f(x) \xder = \sin f(x) + k $

		$ \int \frac{f\prime(x)}{\cos^2 f(x)} \xder = \int f\prime(x) \sec^2 f(x) \xder = 
			\int f\prime(x) [ 1 + \tan^2 f(x) ] \xder = \tan f(x) + k $

		$ \int \frac{f\prime(x)}{\sin^2 f(x)} \xder = \int f\prime(x) \cosec^2 f(x) \xder = 
			\int f\prime(x) [ 1 + \cotan^2 f(x) ] \xder = \cotan f(x) + k $

		$ \int \frac{f\prime(x)}{\sqrt{1 - {[f(x)]}^{2}}} \xder = \arcsin f(x) + k $

		$ \int \frac{-f\prime(x)}{\sqrt{1 - {[f(x)]}^{2}}} \xder = \arccos f(x) + k $

		$ \int \frac{f\prime(x)}{\sqrt{1 + {[f(x)]}^{2}}} \xder = \arctan f(x) + k $

		$ \int f\prime(x) \cosh f(x) \xder = \sinh f(x) + k $

		$ \int f\prime(x) \sinh f(x) \xder = \cosh f(x) + k $

		$ \int \frac{f\prime(x)}{\cosh^2 f(x)} \xder = \tanh f(x) + k $

		$ \int \frac{f\prime(x)}{\sin^2 f(x)} \xder = - coth f(x) + k $

		$ \int \frac{f\prime(x)}{\sqrt{1 + {[f(x)]}^{2}}} \xder = \argsh f(x) + k = 
			\ln | f(x) + \sqrt{1 + {[f(x)]}^{2}} $

		$ \int \frac{f\prime(x)}{\sqrt{{[f(x)]}^{2}} - 1} \xder = \argch f(x) + k = 
			\ln | f(x) + \sqrt{{[f(x)]}^{2} - 1} | + k $

		$ \int \frac{f\prime(x)}{\sqrt{1 - {[f(x)]}^{2}}} \xder = \argth f(x) + k = 
			\ln | f(x) + \frac{1 + f(x)}{1 - f(x)} | + k $

	\section{Integral mugagabeak}
		\subsection{Aldagai aldaketa}
			$ \int f(x) \xder = \left/
				\begin{array}{c}
					x = \phi (t) \\
					dx = \phi'(t) \tder
				\end{array}
				\right/
				= \int f(\phi(t)) \phi' (t) \tder =
				\int g(t) \tder = G (t) + C = G (\phi^{-1} (x)) + C
			$
		\subsection{Zatika}
			$ \int u \, dv = uv - \int v \, du $

			\subsubsection{$ \int R (Pm(x), e^x) \xder$}
				$ u = Pm(x) $ \quad 
				Adb.: $ \int (2x + 1) e^x \xder $
			\subsubsection{$ \int R (Pm(x), \cos ax, \sin bx) \xder $}
				$ u = Pm(x) $ \quad
				Adb.: $ \int (2x + 1) \cos x \xder $
			\subsubsection{$ \int R (e^{\alpha x}, \sin ax, \cos bx) \xder $}
				$ u = e^{\alpha x} $ ó $ u = \sin ax, \cos bx $ \quad
				Adb.: $ \int e^x \cos x \xder $

				Mota hau birritan egiten da, \textit{u} exponentziala ala
				trigonometrikoa izan daiteke; bigarren aldiz egiterakoan
				lehen aukeratutakoa berriro ere aukeratuko beharko genuke.

			\subsubsection{$ \int (\ln, \arcsin, \arccos, \arctan, \argsh, \argch, \argth ) \xder $}
				$ ( \dots ) = u $
		\subsection{Arrazionalak}
			$ \int \frac{P(x)}{Q(x)} \xder =
				\Big{/}
					asdf
				\Big{/}
				= \int [ C(x) + \frac{R(x)}{Q(x)} ] \xder =
				\int C(x) \xder + \int \frac{R(x)}{Q(x)} \xder \leftarrow \text{Urratsak} $

				Urratsak
			\begin{enumerate}
				\item faktorizatu Q(x): $ Q(x) = 0 \Rightarrow
					\Bigg{ \{ } 
						x = a, 
						x = b, 
						\dots,
						x = n 
					\Bigg{ \} } $
				\item deskonposatu $ \frac{R(x)}{Q(x)} $ zatiki simpleetan:
					\begin{itemize}
						\item $ \frac{R(x)}{Q(x)} = \frac{A}{x-a} + \frac{B}{x-b}
							+ \dots + \frac{N}{x-n}
							\begin{array}{c}
								\text{Zatitzaile}\\
								\Rightarrow\\
								\text{komuna}
							\end{array}
							\Bigg \{ %} don't tell your editor about this!
							\begin{array}{c}
								\text{Identifikatu}\\
								\text{edo}\\
								\text{balioak eman}
							\end{array}
							\Rightarrow
							\begin{array}{c}
								\text{Sistema}\\
								\text{ebatzi}
							\end{array} \Rightarrow \text{A, B, } \dots \text{N}$

						\item Soluzio Erreal Anitzak: $ Q(x) = (x-a)^p (x-b)^q $
							$ \frac{P(x)}{Q(x)} = \frac{A_1}{x-a} +
							\frac{A_2}{(x-a)^2} + \frac{A_3}{(x-a)^3} + \dots +
							\frac{A_p}{(x-a)^p} + \frac{B_1}{(x-b)} + \dots +
							\frac{B_q}{(x-b)^q} 
							\begin{array}{c}
								\text{Zatitzaile}\\
								\Rightarrow\\
								\text{komuna}
							\end{array}
							\Bigg \{
							\begin{array}{c}
								\text{Identifikatu}\\
								\text{edo}\\
								\text{balioak eman}
							\end{array}
							\Rightarrow
							\begin{array}{c}
								\text{Sistema}\\
								\text{ebatzi}
							\end{array} \Rightarrow 
							\begin{array}{l}
								A_1, A_2, \dots A_p\\
								B_1, B_2, \dots A_q
							\end{array}$

						\item Soluzio Irudikari Simpleak: $ Q(x) = \dots (a x^2 + b x + c) $
							$ \frac{R(x)}{Q(x)} = \dots +
							\frac{Mx + N}{a x^2 + b x + c}
							\begin{array}{c}
								\text{Zatitzaile}\\
								\Rightarrow\\
								\text{komuna}
							\end{array}
							\Bigg \{
							\begin{array}{c}
								\text{Identifikatu}\\
								\text{edo}\\
								\text{balioak eman}
							\end{array}
							\Rightarrow
							\begin{array}{c}
								\text{Sistema}\\
								\text{ebatzi}
							\end{array} \Rightarrow M, N $
						\item Soluzio Irudikaria Anitzak$ \Rightarrow $ \underline{Hermite}
					\end{itemize} 

				\item Zatiki simpleak integratu
					\begin{itemize}
						\item S.E.S: $ \Rightarrow \int \frac{1}{x-a} \xder = \ln |x-a| + C $
							% SRS xd ( gaztelerazko bertsioan )
						\item S.E.A: $ \Rightarrow \int \frac{1}{(x-a)^n} \xder = \frac{(x-a)^{-n+1}}{-n+1} + C $
						\item S.E.S: $ \Rightarrow \int \frac{Mx+N}{a x^2 + b x + c} \xder = $
					\end{itemize}

			\end{enumerate}
		\subsubsection{Hermite S.I.A}
			$ \int \frac{P(x)}{Q(x)} \xder = \frac{p_2 (x)}{q_2 (x)} + \int \frac{p_1 (x)}{q_1 (x)} \xder $

			$ q_1 (x) $ : $ Q(x)$ren faktoreak osatutako polinomioa

			$ p_1 (x) $ : $ q_1 $ baino gradu bat txikiagoak diren koefiziente indeterminatuen polinomioa

			$ q_2 (x) $ : $ \frac{Q(x)}{q_1 (x)} $

			$ p_2 (x) $ : $ q_2 $ baino gradu bat txikiagoak diren koefiziente indeterminatuen polinomioa

			$ \frac{d}{\xder} \Rightarrow \frac{P(x)}{Q(x)} = \frac{d}{\xder} 
				\Big[ \frac{p_2 (x)}{q_2 (x)} \Big] + \frac{p_1 (x)}{q_2 (x)}
				\begin{array}{c}
					\text{Zatitzaile}\\
					\Rightarrow\\
					\text{komuna}
				\end{array}
				\Bigg \{
				\begin{array}{c}
					\text{Identifikatu}\\
					\text{edo}\\
					\text{balioak eman}
				\end{array}
				\Rightarrow
				\begin{array}{c}
					\text{Sistema}\\
					\text{ebatzi}
				\end{array} \Rightarrow p_1 , p_2 $

				$ \int \frac{p_1 (x)}{q_1 (x)} \xder \Rightarrow $ Zatiki sinpleak

	\subsection{Exponentzialak}
		$ \int R (e^x) \xder $

			aldaketa $ \Rightarrow
			\left/
				\begin{array}{c}
					e^x = t\\
					e^x \xder = \tder
				\end{array}
			\right/
			\Rightarrow $ \underline{Arrazionala}

	\subsection{Irrazionalak}
		\begin{itemize}
			\item $ \int R (x, x^{r/s}, \dots , x^{p/q}) \xder $\\
				aldaketa $ \Rightarrow x = z^{\mu},
				\mu = \text{M.C.M} (s, \dots, q) \Rightarrow Arrazionala $
			\item $ \int R (x, (\frac{ax+b}{cx+d})^{r/s}, \dots, 
				(\frac{ax+b}{cx+d})^{p/q}) \xder $\\
				aldaketa $ \Rightarrow (\frac{ax+b}{cx+d}) = z^{\mu},
				\mu = \text{MKT} (s, \dots, q) \Rightarrow Racional $
			\item aldaketa trigonometrikoa edo hiperbolikoa:\\
				$ \sqrt{a^2 - [f(x)]^2} \Rightarrow f(x) = a \sin t $\\
				$ \sqrt{[f(x)]^2 - a^2} \Rightarrow f(x) = a \cosh t $\\
				$ \sqrt{[f(x)]^2 + a^2} \Rightarrow f(x) = a \sinh t $
			\item kasu bereziak:
				\begin{enumerate}
					\item Metodo Alemaniarra:
						$ \int \frac{P(x)}{\sqrt{a x^2 + bx + c}} \xder =
						q(x) \sqrt{a x^2 + bx + c} + k \int \frac{\xder}{\sqrt{a x^2 + bx + c}} $\\

						$ q(x) $: $ P(x) $ baino gradu bat txikiagoak diren
						koefiziente indeterminatuen polinomioa\\
						$ k $: konstantea
							
						$ \frac{\mathrm{d}}{\xder} \Rightarrow 
						\frac{P(x)}{\sqrt{a x^2 + bx + c}} = 
						\frac{\mathrm{d}}{\xder}
						[q(x) \sqrt{a x^2 + bx + c} + \frac{k}{\sqrt{a x^2 + bx + c}}
						\begin{array}{c}
							\text{Zatitzaile}\\
							\Rightarrow\\
							\text{komuna}
						\end{array}
						\Bigg \{
						\begin{array}{c}
							\text{Identifikatu}\\
							\text{edo}\\
							\text{balioak eman}
						\end{array}
						\Rightarrow
						\begin{array}{c}
							\text{Sistema}\\
							\text{ebatzi}
						\end{array} \Rightarrow q(x), k $
						$ \quad \int \frac{\xder}{\sqrt{a x^2 + bx + c}}
						\Rightarrow
						\begin{array}{c}
							\arcsin\\
							\argsh\\
							\argch
						\end{array} $
					\item Bigarren kasu berezia:
						$ \int \frac{\xder}{(x - \alpha)^n \sqrt{a x^2 + bx + c}} $\\
						aldaketa $ \Rightarrow 
						\begin{array}{c}
							x - \alpha = \frac{1}{z}\\
							\xder = \frac{-1}{z^2} \zder
						\end{array}
						\Rightarrow $ \underline{Arrazionala}
				\end{enumerate}
		\end{itemize}

	\subsection{Binomia: $ \int x^m {(a + b x^n)}^{p} \xder $}
		\begin{enumerate}
			\item $ p \in Z \Rightarrow  
				\text{\underline{Lehenengo motako irrazionala}}
				 \Rightarrow $
				\underline{Arrazionala}
			\item $ \frac{m+1}{n} \in Z \Rightarrow 
				\left/
					\begin{array}{c}
						x^n = t\\
						x = t^{\frac{1}{n}}\\
						\xder = \frac{1}{n} t^{\frac{1}{n}-1} \tder
					\end{array}
				\right/ \Rightarrow
				\int t^{\frac{m}{n}} (a + bt)^p \frac{1}{n}
				t^{\frac{1}{n}-1} \tder = 
				\frac{1}{n} \int t^{\frac{m+1}{n}-1}
				(a + bt)^p \tder =
				\left/
					\begin{array}{c}
						a + bt = z\\
						b \tder = \zder
					\end{array} 
				\right/ = 
				\frac{1}{nb} \int (\frac{z-a}{b})^{\frac{m+1}{n}-1}
				z^p \zder \Rightarrow $
				\underline{Lehenengo motako irrazionala} 
				$ \Rightarrow $
				\underline{Arrazionala}
			\item $ \frac{m+1}{n} + p \in Z \Rightarrow
				\left/
					\begin{array}{c}
						x^n = t\\
						x = t^{\frac{1}{n}}\\
						\xder = \frac{1}{n} t^{\frac{1}{n}-1} \tder
					\end{array}
				\right/ \Rightarrow
				\int t^{\frac{m}{n}} (a + bt)^p \frac{1}{n}
				t^{\frac{1}{n}-1} \tder = \frac{1}{n}
				\int t^{\frac{m+1}{n}-1} (a + bt)^p
				\frac{t^p}{t^p} \tder = \frac{1}{n}
				\int t^{\frac{m+1}{n}-1+p}
				(\frac{a + bt}{t})^p \tder \Rightarrow
				\left/
					\begin{array}{c}
						\frac{a + bt}{t} = z\\
						\frac{-a}{t^2} \tder = \zder
					\end{array}
				\right/ $
				\underline{Lehenengo motako irrazionala}
				$ \Rightarrow $ \underline{Arrazionala}
		\end{enumerate}

	\subsection{Trigonometrikoak}
		\subsubsection{$\int R(\sin x, \cos x) \xder $}
			Mota hau \textit{sin}ua edo \textit{cos}inua zatitzailean\\
			daudenean erabiltzen da, haien potentzien arabera:

			$ \text{(bikoitia):} \tan x = t \quad
				\left\{
				\begin{array}{c}
					\xder = \frac{\tder}{1+{t}^{2}}\\
					\sin x = \frac{t}{\sqrt{1+{t}^{2}}}\\
					\cos x = \frac{1}{1+{t}^{2}}
				\end{array}
				\right.
			$ \quad
			$ \text{(bakoitia):} \tan \frac{x}{2} = t \quad
				\left\{
				\begin{array}{c}
					\xder = \frac{2\tder}{1+{t}^{2}}\\
					\sin x = \frac{2t}{1+{t}^{2}}\\
					\cos x = \frac{1-{t}^{2}}{1+{t}^{2}}
				\end{array}
				\right.
			$
		\subsubsection{$\int R(\tan x) \xder $}
			aldaketa
			$ \left/ \begin{array}{c}
				\tan x = t\\
				\xder = \frac{\tder}{1+{t}^{2}}
			\end{array} \right/ \Rightarrow $ \underline{Arrazionala}

		\subsubsection{$\int \sin^{m} x \times \cos^{n} x \xder$}
			\begin{itemize}
				\item $ m \quad \text{bakoitia} \Rightarrow \cos x = t $
				\item $ n \quad \text{bakoitia} \Rightarrow \sin x = t $
				\item $ 
					\left.
					\begin{array}{c}
						m\\
						n
					\end{array}  \right\} \text{bakoitia} \Rightarrow
					\cos x = t \quad \text{edo}\quad\sin x = t $
				\item $
					\left.
					\begin{array}{c}
						m\\
						n
					\end{array} \right\} \text{bikoitia} \Rightarrow
						\begin{array}{c}
							\cos^{2} x = \frac{1+\cos 2x}{2}\\
							\sin^{2} x = \frac{1-\cos 2x}{2}
						\end{array} $
			\end{itemize}

		\subsubsection{}

			$ \int \cos (mx) \cos (nx) \xder = \frac{1}{2}
					\int [ \cos (mx + nx) + \cos (mx - nx) ] \xder $

			$ \int \sin (mx) \sin (nx) \xder = \frac{1}{2}
					\int [-\cos (mx + nx) + \cos (mx - nx) ] \xder $

			$ \int \cos (mx) \sin (nx) \xder = \frac{1}{2}
					\int [ \sin (nx + mx) + \sin (nx - mx) ] \xder $

	\subsection{Hiperbolikoak}
		\subsubsection{$ \int R(\sinh x, \cosh x) \xder $}
			$ \text{(bikoitia):} \tanh x = t \quad
				\left\{
				\begin{array}{c}
					\xder = \frac{\tder}{1+{t}^{2}}\\
					\sinh x = \frac{t}{\sqrt{1+{t}^{2}}}\\
					\cosh x = \frac{1}{1+{t}^{2}}
				\end{array}
				\right.
			$ \quad
			$ \text{(bakoitia):} \tanh \frac{x}{2} = t \quad
				\left\{
				\begin{array}{c}
					\xder = \frac{2\tder}{1+{t}^{2}}\\
					\sinh x = \frac{2t}{1+{t}^{2}}\\
					\cosh x = \frac{1-{t}^{2}}{1+{t}^{2}}
				\end{array}
				\right.
			$
		\subsubsection{$ \int R(\tanh x) \xder $}
			aldaketa
			$
			\left/
				\begin{array}{c}
					\tanh x = t\\
					\xder = \frac{\tder}{1-{t}^{2}}
				\end{array}
			\right/ \Rightarrow $ \underline{Arrazionala}

		\subsubsection{$ \int \sinh^{m}x \cosh^{n} x \xder $}
			\begin{itemize}
				\item $ m \quad \text{bakoitia} \Rightarrow \cosh x = t $
				\item $ n \quad \text{bakoitia} \Rightarrow \sinh x = t $
				\item $ 
					\left.
					\begin{array}{c}
						m\\
						n
					\end{array}  \right\} \text{bakoitia} \Rightarrow
					\cosh x = t \quad \text{edo}\quad\sinh x = t $
				\item $
					\left.
					\begin{array}{c}
						m\\
						n
					\end{array} \right\} \text{bikoitia} \Rightarrow
						\begin{array}{c}
							\cosh^{2} x = \frac{1+\cosh 2x}{2}\\
							\sinh^{2} x = \frac{\cosh 2x -1}{2}
						\end{array} $
			\end{itemize}
			

\section{Integral Mugatua}
	\subsection{Barrowren erregela}
		$ \int_a^b f(x) \xder = [f(x)]{_a^b} = f(b) - f(a) $
	
	\subsection{Aplikazioak}
		\subsubsection{Azalera}
			$ A = \int_a^b f(x) \xder $

		\subsubsection{Biraketa gorputz baten bolumena}
			$ V_x = \pi \int_a^b y^2 \xder $

			$ V_y = \pi \int_c^d x^2 \xder $

		\subsubsection{Arku baten luzera}
			$ L = \int_a^b \sqrt{1 + {(y\prime)}^{2}} \xder $

		\subsubsection{Biraketa gorputz baten azalera}
			$ A = 2\pi \int_a^b y \sqrt{1 + {(y\prime)}^{2}} \xder $

\end{document}
